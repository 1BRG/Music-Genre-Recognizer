\documentclass[12pt, a4paper]{article}

% --- PACHETE ---
\usepackage[utf8]{inputenc}
\usepackage[T1]{fontenc}
\usepackage[romanian]{babel}
\usepackage{geometry}
\usepackage{amsmath} % Pentru formule matematice
\usepackage{graphicx}
\usepackage{hyperref} % Pentru link-uri (chiar dacă nu sunt, e o practică bună)
\usepackage{listings} % Pentru blocuri de cod
\usepackage{xcolor}   % Pentru culori în listings

% --- CONFIGURĂRI ---
\geometry{a4paper, margin=1in}

% Configurare pentru blocurile de cod (listings)
\definecolor{codegreen}{rgb}{0,0.6,0}
\definecolor{codegray}{rgb}{0.5,0.5,0.5}
\definecolor{codepurple}{rgb}{0.58,0,0.82}
\definecolor{backcolour}{rgb}{0.95,0.95,0.92}

\lstdefinestyle{mystyle}{
    backgroundcolor=\color{backcolour},
    commentstyle=\color{codegreen},
    keywordstyle=\color{magenta},
    numberstyle=\tiny\color{codegray},
    stringstyle=\color{codepurple},
    basicstyle=\ttfamily\footnotesize,
    breakatwhitespace=false,
    breaklines=true,
    captionpos=b,
    keepspaces=true,
    numbers=left,
    numbersep=5pt,
    showspaces=false,
    showstringspaces=false,
    showtabs=false,
    tabsize=2
}
\lstset{style=mystyle}

\lstdefinestyle{shellstyle}{
    backgroundcolor=\color{black},
    basicstyle=\ttfamily\footnotesize\color{white},
    breaklines=true,
    showstringspaces=false,
    frame=tb,
    framerule=0pt,
}


% --- TITLU ---
\title{Documentația Proiectului: \\ Clasificator Naive Bayes Multinomial pentru Predicția Genului Muzical}
\author{Bălăceanu Rafael Gabriel}

% --- ÎNCEPUTUL DOCUMENTULUI ---
\begin{document}

\maketitle
\thispagestyle{empty}
\newpage
\tableofcontents
\thispagestyle{empty}
\newpage
\setcounter{page}{1}

\section{Modelul Matematic: Naive Bayes Multinomial}

Pentru clasificarea genurilor muzicale, formula este:

\begin{equation}
P(\text{Gen} | \text{Versuri}) = {P(\text{Versuri} | \text{Gen}) \times P(\text{Gen})}
\end{equation}

Unde:
\begin{itemize}
    \item \textbf{P(Gen | Versuri)}: probabilitatea ca o piesă să aparțină unui anumit \textit{Gen}, având în vedere \textit{Versurile} sale. Acesta este rezultatul pe care dorim să-l calculăm.
    \item \textbf{P(Versuri | Gen)}: probabilitatea de a întâlni \textit{Versurile} date într-o piesă dintr-un anumit \textit{Gen}.
    \item \textbf{P(Gen)} este probabilitatea a priori: probabilitatea generală ca o piesă să aparțină unui anumit \textit{Gen} în setul nostru de date.
\end{itemize}

\subsection{Ipoteza "Naivă"}

Modelul face o presupunere "naivă" de independență condițională: consideră că prezența fiecărui cuvânt în versuri este independentă de prezența celorlalte cuvinte. Astfel probabilitatea unei piese să aparțina unui anumit gen muzical este probabilitatea fiecarui cuvânt să aparțină acestuia.

\begin{equation}
P(\text{Versuri} | \text{Gen}) \approx P(\text{Gen})\prod_{i=1}^{n} P(\text{cuvânt}_i | \text{Gen})
\end{equation}


\subsection{Probabilități Logaritmice}

Înmulțirea multor probabilități mici (între 0 și 1) poate duce la erori de calcul (underflow numeric). Pentru a evita acest lucru, implementarea calculează suma logaritmilor probabilităților:

\begin{equation}
log(P(\text{Versuri} | \text{Gen})) \approx log(P(\text{Gen})\prod_{i=1}^{n} P(\text{cuvânt}_i | \text{Gen}))
\end{equation}

\begin{equation}
\log(P(\text{Gen} | \text{Versuri})) \approx \log(P(\text{Gen})) + \sum_{i=1}^{n} \log(P(\text{cuvânt}_i | \text{Gen}))
\end{equation}

Genul cu cea mai mare probabilitate logaritmică este ales ca predicție finală.


\subsection{Netezirea Laplace (Aditivă)}

Pentru a gestiona cuvintele care apar în setul de test dar limitate la anumite genuri (ceea ce ar duce la o probabilitate de zero pentru celelalte genuri), se aplică netezirea Laplace. O valoare mica, \textbf{alpha} (setată în cod la 1.0), se adaugă la numărătorul fiecărui cuvânt, prevenind probabilitățile nule.
\section{Structura Codului și Funcțiile Principale}

Proiectul este organizat în trei fișiere Python principale:

\subsection{\texttt{data\_processing.py}}
Acest fișier conține toate funcțiile legate de încărcarea, curățarea și preprocesarea datelor.

\begin{itemize}
    \item \textbf{\texttt{read\_csv(csv\_path, cols, ...)}}: Citește coloanele specificate (\texttt{"Genre"}, \texttt{"Lyrics"}) din fișierul CSV într-un DataFrame pandas și elimină rândurile cu date lipsă.
    \item \textbf{\texttt{tokens\_text(text, ...)}}: Funcția centrală de procesare a textului. Primește textul brut (versurile) și realizează următoarele operațiuni:
    \begin{enumerate}
        \item Elimină caracterele speciale și conținutul din parantezele drepte (ex: "$\left[ Chorus \right]$ ...") specifice textelor provenite din site-ul Genius.
        \item Convertește textul la litere mici.
        \item Elimină punctuația.
        \item Elimină cuvintele comune din limba engleză ("stop words").
        \item Împarte textul curățat într-o listă de cuvinte (token-uri).
    \end{enumerate}
    \item \textbf{\texttt{preprocess\_data(data, column, ...)}}: Aplică funcția \texttt{tokens\_text} pe coloana de versuri a DataFrame-ului și stochează rezultatul într-o nouă coloană, numită \texttt{"Tokens"}.
\end{itemize}

\subsection{\texttt{MultinomialNBayes.py}}
Acest modul conține implementarea clasificatorului.
\begin{itemize}
    \item \textbf{clasa \texttt{MultinomialNaiveBayes}}:
    \begin{itemize}
        \item \textbf{\texttt{\_\_init\_\_(self)}}: Inițializează parametrii modelului: probabilitățile \texttt{a\_priori}, \texttt{vocabularul}, probabilitățile \texttt{condiționate} și factorul de netezire \texttt{alpha}.
        \item \textbf{\texttt{train(self, x\_train, y\_train)}}: Ordonează procesul de antrenare prin apelarea următoarelor două metode:
        \begin{itemize}
            \item \textbf{\texttt{calc\_a\_priori(self, y\_train)}}: Calculează probabilitatea a priori $P(\text{Gen})$ pentru fiecare gen.
            \item \textbf{\texttt{calc\_cond\_voc(self, x\_train, y\_train)}}: Construiește vocabularul și calculează probabilitatea condiționată $P(\text{cuvânt} | \text{Gen})$.
        \end{itemize}
        \item \textbf{\texttt{predict(self, x\_test)}}: Primește o listă de versuri noi și prezice genul pentru fiecare.
        \item \textbf{\texttt{evaluate(self, x\_test, y\_test)}}: Măsoară performanța modelului comparând predicțiile sale cu etichetele reale.
    \end{itemize}
\end{itemize}

\subsection{\texttt{main.py}}
Acesta este scriptul principal care leagă toate componentele.
\begin{itemize}
    \item \textbf{Încărcarea și Preprocesarea Datelor}: Citește fișierul \texttt{Light\_Music\_Dataset.csv} / \texttt{Heavy\_Music\_Dataset1.csv} și preprocesează versurile.
    \item \textbf{Antrenarea și Evaluarea Modelului}:
    \begin{itemize}
        \item \textbf{Optiunea1}: Împarte setul de date într-un set de antrenament (80\%) și unul de test (20\%) folosind
        \texttt{train\_test\_split}.

        \item \textbf{Optiunea2}: Împarte setul de date într-un set de antrenament în care fiecare gen are o anumită proporție din setul original și unul de test, restul datelor rămase, folosind \texttt{create\_false\_imbalance(lyrics, genres, procent\_per\_genre)}.


        \item Inițializează, antrenează și evaluează modelul \texttt{MultinomialNaiveBayes}.
    \end{itemize}
    \item \textbf{\texttt{plot\_confusion\_from\_dict\_proportions(...)}}: Vizualizează rezultatele evaluării sub forma unei matrici de confuzie. (Cod generat cu ChatGPT)
    \item \textbf{\texttt{start\_testing()}}: Inițiază o buclă interactivă în care utilizatorul poate introduce versuri pentru a obține o predicție.
\end{itemize}

\section{Instrucțiuni de Utilizare}
\begin{enumerate}
    \item \textbf{Cerințe preliminare}: Asigurați-vă că aveți Python instalat.
    \item \textbf{Setul de date}: Fișierul folosit \texttt{Light\_Music\_Dataset.csv} sau
    \newline \texttt{Heavy\_Music\_Dataset1.csv} trebuie să se afle într-un folder numit \texttt{Music-Datasets} (sursele fișierelor de date pot fi găsite la începutul fișierului main.py pentru a putea fi descărcate).
    \item \textbf{Instalarea Dependințelor}: Deschideți un terminal și rulați următoarea comandă:
    \begin{lstlisting}[language=bash]
pip install -r requirements.txt
    \end{lstlisting}
    \item \textbf{Rularea Proiectului}: Executați scriptul principal din terminal:
    \begin{lstlisting}[language=bash]
python main.py
    \end{lstlisting}
\end{enumerate}
Scriptul va antrena și evalua mai întâi modelul, afișând rezultatele. Ulterior, va aștepta ca utilizatorul să introducă text pentru clasificare.

\section{Exemplu de Utilizare}
După rularea comenzii \texttt{python main.py}, procesul de antrenare și evaluare va fi afișat în terminal. La final, va porni prompt-ul interactiv.

\subsection{Output în Terminal}
\begin{lstlisting}[style=shellstyle]
Details about Genre:
Metal :  100000
rock :  99997
rap :  99975
pop :  100000
country :  100000
Preprocessing the data...
---------------------------------------------------------------
Training on  399977  datas
Calculating the a priori probability:
country: 0.19959397665365758
Metal: 0.19994399677981484
rock: 0.20009900569282735
pop: 0.20018651072436666
rap: 0.20017651014933358
---------------------------------------------------------------
Calculating the conditioned probability and finding the vocabulary:
Size of vocabulary:  432670
---------------------------------------------------------------
---------------------------------------------------------------
Evaluate the model accuracy on 99995 datas:
60.44102205110256%
---------------------------------------------------------------
\end{lstlisting}
\begin{figure}[h!]
    \centering
    \includegraphics[width=1\linewidth]{Heavy_Dataset_no_imbalance.png}
    \caption{Graficul matricii de confuzie}
    \label{fig:placeholder}
\end{figure}

\subsection{Sesiune Interactivă}
Programul va aștepta acum inputul dumneavoastră pe o singura linie (se poate folosii programul oneLineLyrics.cpp). Pentru a ieși, tastați \texttt{EOF} și apăsați Enter.
\begin{lstlisting}[style=shellstyle]
> Darkness, imprisoning me All that I see, absolute horror
metal

> Country roads, take me home To the place I belong
country

> EOF
\end{lstlisting}

\section{Referințe Bibliografice}
\begin{itemize}
    \item https://www.geeksforgeeks.org/machine-learning/naive-bayes-scratch-implementation-using-python/
    \item LLM-uri precum ChatGPT si Gemini
    \item \href{https://github.com/1BRG/Music-Genre-Recognizer}{Github}
\end{itemize}

\end{document}